\documentclass[t]{beamer}
\usepackage{pgf}
\usepackage{textcomp}
\usepackage{listings}

%\usetheme{EastLansing}
\mode<presentation>
%\useinnertheme{rounded}
\useoutertheme{infolines}
\usecolortheme{spruce}
\setbeamerfont{block title}{size=\large}
\setbeamertemplate{itemize subitem}[triangle]
\setbeamercolor{titlelike}{parent=structure,bg=white!85!MSUgreen}
\mode<all>

\renewcommand{\emph}[1]{{\color{blue} #1}}

\hypersetup{unicode}
\usepackage[utf8]{inputenc}
\usepackage[russian]{babel}

\usepackage{tikz}
\usetikzlibrary{
  positioning,
  fit,
  decorations.pathreplacing,
}

\lstset{language=C, keywordstyle=\bfseries\color{green!40!black},
  commentstyle=\itshape\color{purple!40!black},
  identifierstyle=\color{blue}, stringstyle=\color{orange},
  morekeywords={bool, true, false, uint16_t, size_t, parse_result}}

\title[Введение в программирование]
{
  Введение в программирование\\
  CanSat как автоматическая система
}

\author[Paul Fertser]
{
  Паша Ферцер/Paul Fertser\\
  fercerpav@gmail.com\\
  \url{http://paulfertser.info/cansat-school}
}

\date{Cansat Russia 2015}

\begin{document}

\begin{frame}
  \titlepage
\end{frame}

\begin{frame}[c]{Постановка задачи}
\begin{tikzpicture}
  [block/.style={
      rectangle,
      rounded corners,
      align=center,
      inner sep=0.5cm,
      node distance=2cm,
      draw=red!50!black!50,
      fill=blue!20
    },
  >=stealth, thick]
\node[block] (temp) {t, \textdegree{}C \\ sensor};
\node[block] (press) [below=of temp] {P, kPa \\ sensor};

\node[block] (MCU) [fit=(temp)(press), inner sep=0, text width=4cm,
  left=of temp.north west, anchor=north east, text opacity=0] {MCU
  \\ Микроконтроллер};

\node at (MCU) {?};

\node[block] (RF) [fit=(MCU), inner sep=0, text width=2cm, left=of MCU] {RF
  \\ module};

\draw [<->] (temp) -- node[align=center] {Dallas\\1-wire} (temp.west -| MCU.east);
\draw [->] (press) -- node[auto,swap] {Analog} (press.west -| MCU.east);

\draw [->] (MCU) -- node (txline) {} (RF)
	node[very near start, above] {Tx} node[very near end, above] {Rx};
\node [above of=txline,align=center] {UART};
\draw [->,dashed] ([yshift=-1cm]RF.east) -- ([yshift=-1cm]MCU.west)
	node[very near start, above] {Tx} node[very near end, above] {Rx};

\draw [decorate, decoration={brace, raise=0.5cm, amplitude=0.3cm}] (RF) -- (MCU);
%\draw [->,dashed] (RF.east) +(0,-1) -- ($(MCU.west) + (0,-1)$)
%	node[very near start, above] {Tx} node[very near end, above] {Rx};
\end{tikzpicture}
\end{frame}

\begin{frame}{Автоматические системы}
\begin{itemize}
\item Механические (толкатели, валы, и т.д.)
\item Электрические (концевые выключатели, реле и т.д.)
\item Электронные (на основе радиоламп/транзисторов)
\end{itemize}
\end{frame}

\begin{frame}{Электронные системы}
  \begin{block}{Варианты реализации}
  \begin{itemize}
  \item Дискретные компоненты
  \item ASIC
  \item FPGA/CPLD (ПЛИС)
  \item Универсальный вычислитель (компьютер)
  \end{itemize}
  \end{block}
\end{frame}

\begin{frame}{История вычислительных устройств}
  \begin{block}{$\lambda$-исчисление}
    Разработано Алонсо Чёрчем для решения вопроса о <<механической
    вычислимости>>

    Языки программирования: LISP, Erlang, OCaml, Haskell
  \end{block}
  \begin{block}{Универсальная машина Тьюринга}
    Первая концепция компьютера с хранимыми программами (выполняет
    разные вычисления в зависимости от записанной программы)

    Языки программирования: ассемблеры, C, COBOL, Pascal
  \end{block}

  \begin{block}{}
    \begin{center}
      Эквивалентны с математической точки зрения
    \end{center}
  \end{block}
\end{frame}

\begin{frame}[fragile]{Функции}
В <<математическом смысле>> (чистые, без побочных эффектов):
$$
f(a,b) = a^2 + b + 1
$$

В <<компьютерном смысле>>:
\begin{lstlisting}
int printsum(int a, int b)
{
  int c = a + b;
  printf("%d", c);
  return c;
}
\end{lstlisting}
\end{frame}

\begin{frame}{Создание функций}
  \begin{block}{Вам нужна отдельная функция если:}
    \begin{itemize}
    \item этот код нужен больше, чем в одном месте
    \item исходная задача описывает её отдельно
    \item имеющаяся функция слишком длинная (> $\approx$25 строк)
    \item вычисления можно отделить от побочных эффектов
    \end{itemize}
  \end{block}

  \begin{block}{Рекомендации}
Имя функции должно явно указывать на то, что она делает, и есть ли у
неё побочные эффекты. Для <<чистых>> (не производящих эффекты и не
зависящих от глобальных переменных) функций пишите unit-test'ы.
  \end{block}
\end{frame}

\begin{frame}{Практическая реализация}
  \begin{columns}
    \column{.5\textwidth}
    \begin{block}{$\lambda$-исчисление}
      \begin{itemize}
      \item оперирует функциями без побочных эффектов
      \item[+] возможность математически строгой верификации
      \item[+] элегантные решения
      \item[---] производительность
      \item[---] сложность применения на реальных устройствах
      \end{itemize}
    \end{block}

    \column{.5\textwidth}
    \begin{block}{Машина Тьюринга}
      \begin{itemize}
      \item функции с побочными эффектами
      \item [+] близость к аппаратному обеспечению
      \item [+] наличие большого количества инструментов
      \item [---] сложности с обеспечением надёжности
      \item [---] более запутанный исходный код
      \end{itemize}
    \end{block}
    
  \end{columns}
\end{frame}

\begin{frame}{Симуляция}
\Large
Симуляция микроконтроллеров:

  \begin{itemize}
  \item VMLAB (\url{http://www.amctools.com})
  \item HAPSIM + AVR Studio 4 (\url{http://www.helmix.at/hapsim})
  \item Proteus VSM (\$)
  \end{itemize}

Симуляция схем:

  \begin{itemize}
  \item Qucs (\url{http://qucs.sf.net})
  \item Multisim (\$)
  \end{itemize}
\end{frame}

\begin{frame}{Переменные в C}
  Любая переменная имеет:
  \begin{itemize}
  \item имя
  \item тип, размер
  \item адрес
  \item область видимости
    \begin{itemize}
    \item локальная
    \item глобальная
    \end{itemize}
  \item время жизни (только для локальных переменных)
    \begin{itemize}
    \item автоматическая (по умолчанию)
    \item статическая
    \end{itemize}
  \end{itemize}
\end{frame}

\begin{frame}{Булевые операции}
\Large
  \begin{itemize}
  \item \emph{AND} (\emph{\&} (\emph{\&\&} для лог. значений),
    \emph{$\land$}, конъюнкция)
  \item \emph{OR} (\emph{|} (\emph{||} для лог. значений),
    \emph{$\lor$}, дизъюнкция)
  \item \emph{NOT} (\emph{\~{}} (\emph{!} для лог. значений), \emph{$\lnot$}, отрицание)
  \end{itemize}
\end{frame}

\begin{frame}{Управляющие конструкции в языке C}
  \begin{itemize}
  \item \emph{\{\}} --- объединение нескольких действий в блок (действие)\\
    \lstinline{\{ <action1>; <action2>; \}}
  \item \emph{if} --- условное выполнение\\
    \lstinline{if (<condition>) <action1>; [else <action2>;]}
  \item \emph{while} --- оператор цикла\\
    \lstinline{while (<condition>) <action>;}
  \item \emph{for} --- оператор цикла\\
    \lstinline{for (<action1>; <condition>; <action2>) <action3>;}
  \item \emph{do-while} --- оператор цикла\\
    \lstinline{do <action>; while (<condition>);}
  \end{itemize}
\end{frame}

\begin{frame}[fragile]{Пример программы на C99, AVR}
\begin{lstlisting}
...
#define LED_PIN 4
...
void init_hardware(void)
{
  DDRA |= 1 << LED_PIN;
  PORTA |= 1 << LED_PIN;
  _delay_ms(1000);
  PORTA &= ~(1 << LED_PIN);
  ...
  UCSR0B |= (1<<RXEN) | (1<<UDRIE0);
  ...
}
...
\end{lstlisting}
\end{frame}

\begin{frame}[fragile]{Пример программы...}
\begin{lstlisting}
...
int main(void)
{
  init_hardware();
  while (1) {
    if (buf_ready) {
      enum parse_result r =
        parse_packet(buf, sizeof(buf));
      buf_ready = false;
      execute_action(r);
    }
    ...
  }
  return 0;
}
\end{lstlisting}
\end{frame}

\begin{frame}[fragile]{Пример программы...}
\begin{lstlisting}
#define BUF_SIZE 16
volatile char buf[BUF_SIZE];
size_t buf_idx;
volatile bool buf_ready;
...
ISR(USART0_RX_vect)
{
  buf_ready = false;
  buf[buf_idx++] = UDR0;
  if (buf_idx == sizeof(buf)) {
    buf_ready = true;
    buf_idx = 0;
  }
}
...
\end{lstlisting}
\end{frame}

\begin{frame}[fragile]{Пример программы...}
\begin{lstlisting}
...
enum parse_result parse_packet(const char b[],
                               size_t size)
{
  uint16_t crc = crc16(b, size);
  if (crc != (b[0] & (b[1] << 8)))
    return CRC_ERROR;
  switch (b[2]) {
    ...
    case 'S': return START_ENGINE;
    ...
    default: return UNKNOWN_PACKET;
  }
}
...
\end{lstlisting}
\end{frame}

\begin{frame}{Увеличение надёжности императивных программ}
  \begin{block}{Возможные способы}
  \begin{itemize}
  \item отделять чистые функции от побочных эффектов
  \item выполнять требования MISRA C
  \item применять статические анализаторы кода (Cppcheck, Coverity)
  \item использовать языки со строгой типизацией (напр, C++, Ada)
  \end{itemize}
  \end{block}
\end{frame}

\begin{frame}{Литература}
\Large

\url{http://paulfertser.info/c1year.shtml}

  \begin{itemize}
  \item В. Подбельский, С. Фомин
  \item Керниган, Ричи
  \item \url{http://forum.easyelectronics.ru}
  \end{itemize}
\end{frame}

\begin{frame}{Домашнее задание}
\Large
  Написать программу, считывающую с клавиатуры 32-битное беззнаковое
  число \emph{n}, числа \emph{a} и \emph{b} и выводящую на экран
  число, равное исходному \emph{n}, но с установленным битом номер
  \emph{a} и снятым битом номер \emph{b}.
  \vfill

  Разобраться во всех элементах синтаксиса C99 и AVR libc,
  использованных в примере программы в этой презентации. Найти и
  описать возможные последствия от имеющегося в примере \emph{race
    condition}.
\end{frame}
\end{document}
